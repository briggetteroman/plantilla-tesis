%%%%%%%%%%%%%%%%%%%%%%%%%%%%%%%
%%%Autor: Briggette Román
%%%%%%%%%%%%%%%%%%%%%%%%%%%%%%%


\documentclass[12pt,a4paper]{book}
%%%%%%%%%%%%%%%%%
%Manejo de paquetes
%%%%%%%%%%%%%%%%%
\usepackage[utf8]{inputenc} 
\usepackage[spanish]{babel} 
\usepackage{hyperref}
\usepackage{algorithm}
\usepackage{algorithmic}
\input{spanishAlgorithmic} % mi archivo de traducción
\usepackage{titling} %Para llamar los datos
\usepackage{titlesec} 
\usepackage{lipsum} %Muestras Lorem Ipsum
\usepackage{tikz}
\usepackage{afterpage}
\newcommand\blankpage{%
	\null
	\thispagestyle{empty}%
	\addtocounter{page}{0}%
	\newpage}
\usepackage{listings}
\renewcommand{\lstlistingname}{Código}
\renewcommand\lstlistlistingname{Índice de Código}
\usepackage{acronym}
%\acsetup{list-style=tabular}

%%%%%%%%%%%%%%%%%%
%%%% Datos de la universidad, tesista y asesor
\title{{Esquema de Monitorización en Seguridad para entorno IoT }}
\author{Briggette Olenka Román Huaytalla}
\def\supervisor{Mg. Gipsy Miguel Arrunategui Angulo}
\def\university{\href{http://www.uni.edu.pe/}{Universidad Nacional de Ingeniería}}
\def\faculty{Facultad de Ciencias}
\def\school{Escuela Profesional de Ciencia de la Computación}

%%%%%%%%%%%%%%%%%%%
%%%%%%%%%%%%%%%%%%%
%%%%Inicio de archivo
\begin{document}

%%%%%%%%%%%%%%%%%%%
%%% Caratula

	\begin{titlepage}
		\begin{center}
		
		{\scshape \huge \university \par}
		\vspace{0.2cm}
		{\scshape \Large \faculty \par}
		\vspace{0.2cm}
		{\scshape \large \school \par}
%		\vspace{0.2cm}

		\begin{figure}[h]
			\centering
			\includegraphics[scale = 0.5]{figure/log_uni}
		\end{figure}
%		\vspace{0.2cm}
		
		{\LARGE \thetitle}\\[1cm]
		{\Large \textbf{SEMINARIO DE TESIS II}}\\[0.5cm]
		\large\emph{Autor: \\[0.3cm]}
		{\theauthor}\\[0.5cm]
		\large\emph{Asesor:\\[0.3cm] }
		{\supervisor }\\[1cm]
		{\large Diciembre, 2018}\\[4cm] 
		\end{center}
			
	\end{titlepage}
\afterpage{\blankpage}
%%%%%%%%%%%%%%%%%%%%

\frontmatter

%%%%%%%%%%%%%%%%%%%%

%%%%%%%%%%%%%%%%%%%%%%%
%%% Estilo de capitulos

\titleformat{\chapter}[display]
{\Large\itshape}
{\filright\MakeUppercase{\chaptertitlename} }
{10ex}
{\titlerule\vspace{1ex}\filleft}
[\vspace{1ex}\titlerule]

%%%%%%%%%%%%%%%%%%%%%%%

%%%%%%%%%%%%%%%%%%%%%%%
%%% Resumen

\chapter{Resumen}
	
		\lipsum[1] 
	
	\vspace{10px}	
	\textbf{PALABRAS CLAVE:} IoT, Ciberseguridad, OWASP, vulnerabilidades, Gestión de Riesgos.
	
	\vspace{100px}
	
	\textit{El hardware es fácil de proteger: bloquearlo en una habitación, asegurarlo en un escritorio, o comprar uno de repuesto. La información posee más de un problema. Puede existir en más de un lugar, ser transportada a medio planeta en segundos, y ser robada sin que uno tenga conocimiento.}
	
	\begin{flushright}
		(Bruce Schneider, Protege tu Macintosh, 1994)
	\end{flushright}

%%%%%%%%%%%%%%%%%%%%%%%%%
%%% Indices y Listas

\afterpage{\blankpage}
\tableofcontents
\afterpage{\blankpage}
\listoffigures
\afterpage{\blankpage}
\listoftables
\afterpage{\blankpage}
\lstlistoflistings
\afterpage{\blankpage}

%%%%%%%%%%%%%%%%%%%%%%%%%%

%%%%%%%%%%%%%%%%%%%%%%%%%%
%%%Lista de acronimos

\newpage
\chapter*{Lista de Acrónimos}
\begin{acronym}[MMMMMPC] %Ayuda a alinear la lista

\acro{OWASP}{Open Web Application Security Project} \\
\acro{IoT}{Internet of Things}\\
\acro{RPi}{Raspberry Pi} \\
\end{acronym}
\afterpage{\blankpage}

%%%%%%%%%%%%%%%%%%%%%%%%%%

%%%%%%%%%%%%%%%%%%%%%%%%%%
%%% Agradecimientos

\chapter*{Agradecimientos}
\lipsum[1]
\afterpage{\blankpage}

%%%%%%%%%%%%%%%%%%%%%%%%%%
%%%%%%%%%%%%%%%%%%%%%%%%%%
%%% Formato de capitulos
	
\titleformat{\chapter}[display]
{\bfseries\Large}
{\filright\MakeUppercase{\chaptertitlename} \Huge\thechapter}
{1ex}
{\titlerule\vspace{1ex}\filleft}
[\vspace{1ex}\titlerule]	
%%%%%%%%%%%%%%%%%%%%%%%%%%

%%%%%%%%%%%%%%%%%%%%%%%%%%
\mainmatter
%%%%%%%%%%%%%%%%%%%%%%%%%%

%%%%%%%%%%%%%%%%%%%%%%%%%%
%%% Capitulos
\chapter{Introducción}
\lipsum
\section{title}
\subsection{title}
\lipsum	
\chapter{Estado del arte}
\lipsum[1-2]
adasdad \ac{OWASP}

\section{title}
\lipsum[1-3]
\subsection{title}
\lipsum[1] \cite{ISO27005}\cite{NISTGaithersburg2012}.


%%%%%%%%%%%%%%%%%%%%%%%%%%
	
\end{document}